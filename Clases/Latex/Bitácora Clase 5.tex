\documentclass [letterpaper, 12 pts, oneside]{article} 
\usepackage[utf8]{inputenc}
\title{\Huge Taller de Herramientas Computacionales}
\author{Stephanie Escobar Sánchez}
\date{14-enero-2019}

\begin{document}
	\maketitle

	\newpage
	
	\title{\Huge Bitácora clase 5} \\
	\\
	En esta clase comenzamos viendo nuevos comandos de python y formas de agragar información, poner el símbolo "gato" para poner un comentario y las comillas sirven para hacer cadenas, texto que no afecta a las cuentas ''' con ellas se puede anexar texto multilinea. /n es un caracter de salto de linea una cadena en python es un grupo de caracteres que estan delimitados por una comilla al principio y al final.\\
	\\
	El caracter "Porcentaje" indica el lugar en donde se va a sustituir el valor de una variable y con .4f son floats y es cuantos flotantes queremos que nos muestre, si queremos 2 es "porcentaje".2f \\
	Con "porcentaje" y:\\
	\\
	-g indicamos la posición en donde va a poner la primer variable \\
	-s Indicamos que lo va a sustituir por una variable que contenga una cadena \\
	-f es el número de flotantes que queremos poner \\
	-e es para poner exponencial\\
	-d es un numero entero\\
	\\

	Python también tiene modulos, un módulo es una biblioteca donde están definidas una serie de funciones que se necesitan importar\\ 
	Las palabras reservadas o comandos tienen un significado dentro de un lenguage de programación \\
	Import es la palabra reservada para importar los módulos\\
	
	Cuando queremos poner una formula de un modulo se pone el nombre del módulo.lafunción, por ejemplo el módulo math nos permite hacer operaciones matemáticas más avanzadas, como las raices. Para sacar la raíz de 9 se pone\\
	math.squrt (9)\\
	
	Para definir se utiliza def lo que nos permite definir nuestras variables y crear nuestros propios módulos.\\
	\\
	Al final de la clase vimos una breve introducción a LateX, que nos permite crear documentos, utilizamos la herramienta TeXstudio para fines didácticos y aprendimos comándos básicos de iniciar un documento, especificar el tipo, tamaño, etc. Y además vimos que también se pueden importar paquetes para tener funciones específicas.
	
		
\end{document}
	
	
	
	
	