\documentclass [letterpaper, 12 pts, oneside]{article} 
\usepackage[utf8]{inputenc}
\title{\Huge Taller de Herramientas Computacionales}
\author{Stephanie Escobar Sánchez}
\date{14-enero-2019}

\begin{document}
	\maketitle

	\newpage
	
	\title{\Huge Cuestionario} \\
	\\
	¿Qué es un sistema operativo?\\
	R: Es un onjunto de instrucciones y programas que controlan los procesos básicos de una computadora, nos permite conectar el software con el hardware.\\
	\\
	¿Qué es un kernel?\\
	R: Un núcleo o kernel es una parte fundamental del sistema operativo, es el núcleo en donde está compilado y a partir de ahí es que se desarrollan otros sistemas.\\
	\\
	¿Qué es linux?\\
	R: Linux es el kernel de los sistemas operativos que se derivan de el.\\
	\\
	¿Qué son las distribuciones?\\
	R:Son sistemas operativos cuyo núcleo es lunux y a partir de ahí derivan funciones específicas. Al tener el mismo núcleo no difieren tanto y hay distribuciones que derivan de otras.\\
	\\
	¿Qué es un shell?\\
	R: Un shell es un intérprete de comandos, el puente entre el usuario y la computadora.\\
	\\
	¿Qué es un comando?\\
	R: Es una palabra específica que permite dar una instrucción en un lenguaje de programación\\
	\\
	¿Qué es un bit?\\
	R: Es una unidad de información en codigo binario. \\
	\\
	¿Cual es la diferencia entre 32 y 64 bits?\\
	R: Una computadora con 64 bits tiene una mayor capacidad de información, por lo que es más rápida al realizar alguna actividad.\\
	\\
	¿Cómo puedo ver cómo funciona un comando?\\
	R: Escribiendo man y el comando que necesitas saberr te manda al manual de ese comando\\
	\\
	¿Cómo funcionan los permisos en Linux?\\
	R: Hay 3 tipos de permiso: De lectura, de escrutura y de ejecución. Además se puede especificar quien tiene cada permiso, todo el púbico, el usuario o unicamente root.\\
	\\ 
	¿Qué es un git?\\
	R: Es un software de control de versiones que nos permite trabajar en equipo, o subir programas e interactuar con otros usuarios. Así como tener archivos guardados en la nube.
	\\
	\\
	¿Qué es github?\\
	R:Es un servidor de git que te permite tener acceso a la información por medio de una cuenta.\\
	\\
	¿Por qué es importante el pull y el push en git?\\
	R: Debido a que así podemos tener las actualizaciones más recientes de nuestro repositorio y trabajar sobre la versión más nueva sin perder información.\\
	\\  
	¿Qué es un directorio?\\
	R: Es una carpeta en dónde puede haber guardados más directorios y distintos tipos de archivos.\\
	\\
	¿Que son los archivos ocultos?\\
	R: Son archivos que comienzan con un punto, en general son archivos importantes que no queremos que todos vean o configuraciones del sistema que no se deben borrar.\\
	\\
	¿Qué es vi?\\
	R: Es un editor de textos que nos permite crear archivos desde la terminal y en cualquier lenguaje.\\
	\\
	¿Cómo se crea un directorio?\\
	R: Con el comando mkdir, o directamente en la carpeta donde se quiere crear\\
	\\
	¿Qué es un IDE?\\
	R: un entorno de programacion integrado, está desarrollado para el aprendizaje en python. Tiene un editor especializado, una forma de comunicarse con el intérprete e incorpora una herramienta para la depuración \\
	\\
	¿Qué es python?\\
	R:Es un lenguaje de programación que te permite crear programas para resolver problemas específicos. Se usa principalmente en el ámbito científico. \\
	\\
	¿Por qué utilizamos la versión 2 de Python?\\
	R: Debido a que al ser más antigua está mejor desarrollada y hay más paquetes ya hechos\\
	\\
	¿Qué es una variable en python?\\
	R: Es una letra o palabra a la que se le asigna un valor\\
	\\
	¿Qué es un módulo?\\
	R: Es una biblioteca donde estan definidas una serie de funciones que se necesitan importar \\
	\\
	¿Cuales son las formas de poner comentarios en Python?\\
	R:Con un signo de gato o con comillas\\
	\\
	¿Cuál es el equivalente del apt de ubuntu en fedora?\\
	R: dnf\\
	\\
	¿Para qué sirve el símbolo de porcentaje en Python?\\
	R: Para indicar el lugar en donde se va a sustituir un valor o el lugar en donde se va  aplicar la instrucción.\\
	\\
	¿Qué módulo de python nos permite hacer operaciones matemáticas?\\
	R: Math\\
	\\
	¿Cómo se importa un paquete en python?\\
	R: Con el comando import y después el paquete\\
	\\
	¿Qué es Latex?\\
	Es un lenguaje de programación que nos permite hacer archivos de texto con formato.\\
	\\
	¿Para qué sirve TeXstudio?\\
	R: Es una herramienta que te permite crear archivos de latex con una interfaz gráfica más amigable.\\
	\\
	
\end{document}
