\documentclass{book}
\usepackage [spanish]{babel}
\usepackage[utf8]{inputenc}
\usepackage{hyperref}
\usepackage{biblatex}
\title{Taller de Herramientas Computacionales}
\author{Stephanie Escobar sánchez}
\date{17-enero-19}

\begin{document}
	\maketitle
	%Aquí inicia el índice de contenido del texto
	\tableofcontents
	\section*{Introducción} Este libro es para fortalecer los conocimientos de la materia THC\\
	\url{www.google.com}\\
	\hyperref[Google]{www.google.com}

	\chapter{Uso básico de Linux}
	\section{Distribuciones de Linux}
	\section{Comandos}
	\chapter{Introducción a Latex}
	\chapter{Introducción a Python}
	\begin{verbatim}


	#!/usr/bin/python2.7
	# -*- coding: utf- 8 -*-
	
	print 'Hoy es miércoles'
	'''
	Stephanie Escobar Sánchez, 311210666
	Taller de herramientas computacionales
	Este es un programa que dice 'Hoy es miércoles'
	
	'''
	
	x = 10.5; y = 1.0/3; z = 15
	#x, y, z = 10.5,1.0,15.3 (otra forma)
	
	H= """
	El punto en R3 es:
	(x, y, z) = %.2f, %g, %G)
	""" % (x, y, z)
	print H
	
	G= """
	El punto en R3 es:
	(x, y, z) = ({laX: .2f}, {laY:g}, {laZ:G})
	""" .format (laX=x, laY=y, laZ=z)
	print G
	
	import math as m
	from math import sqrt
	from math import sqrt as s
	x = input("¿Cuál es el valor al que le quieres calcular la raíz cuadrada?")
	print "la raiz cuadrada de %.2f es es %f" % (x, m.sqrt(x))
	print sqrt (4)
	print s(16.5)
		\end{verbatim}
		\input{/home/stephanie/Escritorio/THC/Taller-de-Herramientas-Computacionales/Clases/Latex/Prueba.py}
		%solo si está en la misma ruta
	\section*{Orientación a objetos}
	
	\begin{thebibliography}{9}
\bibitem{libro}
\textit{cualquier cosa}
		Author bla bla, 2019
	
	\end{thebibliography}
	
\end{document}