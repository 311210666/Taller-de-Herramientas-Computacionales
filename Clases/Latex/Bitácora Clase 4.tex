\documentclass [letterpaper, 12 pts, oneside]{article} 
\usepackage[utf8]{inputenc}
\title{\Huge Taller de Herramientas Computacionales}
\author{Stephanie Escobar Sánchez}
\date{14-enero-2019}

\begin{document}
	\maketitle

	\newpage
	
	\title{\Huge Bitácora clase 4} \\
	\\
	Comenzamos la clase conociendo un poco más sobre python, para ello era importante tenerlo en la computadora, en el curso utilizamos python 2 debido a que ya hay muchos más paquetes desarrollados en éste que en python 3 ya que es muy nuevo. Para el caso de quienes llevamos nuestra computadora era importante saber qué versión tenemos por medio del comando \textit{-- version}, en mi caso solo tenía la versión 3.6, por lo que instalamos la versión 2.7. \\
	\\
	Para facilitar el uso de python con fines didácticos utilizamos idle-python, idle es un IDE (entorno de programacion integrado) esta desarrollado para el aprendizaje en python. Tiene un editor especializado, una forma de comunicarse con el intérprete e incorpora una herramienta para la depuración.\\
	\\
	Ya trabajando en python hicimos un problema de velocidad y aceleración muy sencillo, en el que primero tuvimos que pensar y resolver en el pizarrón, después de tener el resultado utilizamos python unicamente como una calculadora y aprendimos algunas cosas de inicio:\\
	\\
	-Python solo toma números enteros en las divisiones, por ello es importante que al dividir al menos uno de los dos tenga un decimal, aunque sea un .0.\\
	-Las operaciones básicas de Python son +,-,*,/.\\
	-Para elevar a una potencia se ponen dos asteriscos **\\
	-Para que python te de el resultado es necesario poner el comando \textit{print}\\
	\\
	Finalmente, después de poner únicamente el problema con números vimos las variables, se puede guardar un valor a una variable. Así que unicamente asignamos un valor a cada variable de la fórmula y pusimos la fórmula a correr. \\
	 
	
	
	
\end{document}