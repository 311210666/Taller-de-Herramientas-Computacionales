\documentclass{article}
\usepackage{amsmath}
\usepackage{amssymb}
\usepackage{graphicx}
\usepackage{enumitem}
\usepackage[utf8]{inputenc}
\usepackage{xcolor}

\graphicspath{{/home/stephanie/Escritorio/THC/Taller-de-Herramientas-Computacionales/Clases/Latex/Imagenes/}}

\title{\Huge Taller de Herramientas Computacionales}
\author{Stephanie Escobar Sánchez}
\date{24/enero/2019}


\begin{document}
	\maketitle
	\begin{center}
		\includegraphics[scale=0.40]{1.png}	
	\end{center}
	\newpage
	\begin{center}
		\title {\textcolor{orange}{\Huge \textbf{Cuestionario clase 14}} }  
	\end{center}
\textbf{¿Cómo se puede representar una matriz?}\\
Con una lista de listas\\
\\
\textbf{¿Qué comandos nos permiten dar condiciones para ejecutar una acción?}\\
Los comandos \textit{if y else}\\
\\
\textbf{¿Para qué sirve el comando \textit{elif}?}\\
Para escribir directo la condición si es lo que sigue despuŕs de un \textit{else}\\
\\
\textbf{¿Cual es la diferencia entre una lista y una tupla?}\\
La diferencia entre una lista y una tupla es que una tupla es una lista constante, para definirla se usan paréntesis en vez de corchetes, no se puede modificar. La lista si se puede modificar y se usan corchetes.\\
\\
\textbf{¿Para qué sirve la librería numpy?}\\
Tiene métodos de asuntos científicos\\
\\
\textbf{¿Para qué sirve \textit{pip}?}\\
pip sirve para instalar un modulo de python\\
\\

\end{document}
