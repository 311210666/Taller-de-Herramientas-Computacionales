\documentclass{article}
\usepackage{amsmath}
\usepackage{amssymb}
\usepackage{graphicx}
\usepackage{enumitem}
\usepackage[utf8]{inputenc}
\graphicspath{{/home/stephanie/Escritorio/THC/Taller-de-Herramientas-Computacionales/Clases/Latex/Imagenes/}}

\title{\Huge Taller de Herramientas Computacionales}
\author{Stephanie Escobar Sánchez}
\date{22/enero/2019}

\begin{document}
	\maketitle
	
Tarea: Fibonacci son 2 (guardar la memoria de la lista)
pablo barrera
Tupla:lista constante, para definirla se usan paréntesis en vez de corchetes, no se puede modificar. 
Para laberinto, renglón columna
numpy tiene métodos de cosas científicas 
elif te ahorra un bloque 
Laberinto de abejas 
pip para instalar un modulo de python pip install 'modulo de python' y se puede instalar
Para cadenas de ADN:
si se ponen las cadenas de adn 
operadosr de autoincremento 
incrementa la cantidad que yo pongo a la derecha el valor de una variable

	
\end{document}