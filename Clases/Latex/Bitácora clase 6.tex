\documentclass{article}
\usepackage{amsmath}
\usepackage{amssymb}
\usepackage{graphicx}
\usepackage{enumitem}
\usepackage[utf8]{inputenc}
\graphicspath{{/home/stephanie/Escritorio/THC/Taller-de-Herramientas-Computacionales/Clases/Latex/Imagenes/}}

\title{\Huge Taller de Herramientas Computacionales}
\author{Stephanie Escobar Sánchez}
\date{15/enero/2019}
\begin{document}
	\maketitle
\begin{center}
	\includegraphics[scale=0.40]{1.png}	
\end{center}
\newpage
\title{\Huge Bitácora clase 6} \\
\\
La clase comenzó viendo las diferencias entre virtualización y emulación, debido a que no todos tiene una maquina con sistema operativo linux y optaron por una máquina virtual. Una máquina virtual es más rápida que un emulador debido a que un emulador solo imita lso procesos pero en una maquina virtual es posible incluso instalar programas del sistema operativo que se virtualiza.\\
Otro concepto que vimos fue de bios, que es una memoria que contiene la informacion de la computadora: Procesador, memoria, características específicas del equipo.\\
\\
Hablamos sobre las distintas distribuciones de Linux adaptadas a necesidades específicas y tipo de computadora, las distribuciones tienen distintas versiones llamadas flavors y algunas de ellas son más ligeras que otras, lo cual permite que corran más rápido en computadoras con menor capacidad. El comando \$ top permite ver los núcleos de la computadora para ver la capacidad que tiene y decidir el sistema operativo. Nosotros utilizamos ubuntu y fedora por la cantidad de funciones que tienen y porque son sistemas fáciles de instala.\\
\\
Un algoritmo es un proceso de instrucciones finitas, en la clase aprendimos a hacer algoritmos. Primero planteando un problema y pensar en la solución de forma teórica. El problema de la clase fue ¿Cómo se define la raíz cuadrada? teníamos que crear una función que nos regresara la raíz cuadrada de un número para ello necesitábamos llegar a una aproximación debido a que la computadora no tiene a todos los números reales, por lo que hay un número mínimo que puede darnos.
Pensamos el problema a partir de un cuadrado, sabemos que un cuadrado con lado $\sqrt{x}$ tendrá por área x, sabemos también que un rectángulo con lados x y 1 tiene área x. Primero debemos buscar una forma de pasar del segundo cuadrado al primero.
\begin{center}
	\includegraphics[scale=0.40]{2.png}	
\end{center}
Lo primero que hacemos es pensar en ir cambiando los valores a los lados hasta que su diferencia sea muy pequeña, por tanto primero hacemos un promedio muy lo llamamos h, después esa h se la restamos al otro número.\\
\\
Vimos otros conceptos y comandos de python. Una asignación consiste en asignar valores a una variable a partir de una función o directamente.
En python son importantes los bloques debido a que 
Bloque: todo lo que esta indexado. todo lo que esta después de los dos puntos
Echo: que me mjuestre el resultado de lo que hice
Spline, por cada 4 puntos se construye una función

If es solo una vez, si yo quiero repetir algo es con while 
while print hace todo fuera del bloque

\end{document}
