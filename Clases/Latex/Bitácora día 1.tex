\documentclass [letterpaper, 12 pts, oneside]{article}
\usepackage[utf8]{inputenc}
\title{\Huge Taller de Herramientas Computacionales}
\author{Stephanie Escobar Sánchez}
\date{14-enero-2019}

\begin{document}
	\maketitle

	\newpage
	
	\title{\Huge Bitácora clase 1} \\
	
	El primer día hablamos sobre los objetivos del curso, la forma en la que trabajaríamos todos desde cero hasta poder formar programas. Hablamos de los temas que vamos a cubrir y de la forma de evaluación.\\
	Aprendimos qué es un sistema operativo y además de las distintas distribuciones de Linux, entre las que destacan:Debian, Ubuntu, fedora y linux mint.\\
	Además vimos distintos comandos en la terminal y su utilidad. Los comandos que vimos fueron los siguientes: \\
	\\
	\textit{uname -a:} Te permite ver la información del kernel, nombre, versión, y además es un método para saber si nuestra computadora es de 32 o 64 bits. \\
	\\
	\textit{ls -l bin/bash:} te da información de los permisos\\
	\\
	\textit{touch/tmp/"archivo":} Crea un archivo nuevo\\
	\\
	\textit{ls -1 /tmp/"archivo"}  muestra el archivo que creamos\\
	\\
	\textit{chmod 700 /tmp/"aerchivo":} Cambia los permisos del archivo, existen distintos tipos de permisos, de lectura, de escritura y de ejecución. Este comando nos permite cambiar los permisos a conveniencia. \\
	\\
	\textit{/tmp/"archivo":} Lo ejecuta\\
	\\
	\textit{echo:}muestra el contenido de una variable de estado\\
	\\
	\textit{set:} Muestra las variables de estado\\
	\\
	\textit{pwd:} Indica en que directorio estoy\\
	\\
	\textit{cd:} Me permite regresar a home\\
	\\
	\textit{cd /"un directorio":} Me permite cambiar de directorio.\\
	\\
	\textit{man:} Nos permite acceder al manual de los comandos.\\
	\\
	\\
	Finalmente hablamos sobre la importancia de la programación en carreras como las matemáticas aplicadas y sobre su gran utilidad en distintas áreas, todos los comandos que vimos fueron únicamente téoricos, ya que no teniamos acceso a computadores ese día. Aún así tuvimos una introducción a la programación.
		 
\end{document}