\documentclass{article}
\usepackage{amsmath}
\usepackage{amssymb}
\usepackage{graphicx}
\usepackage{enumitem}
\usepackage[utf8]{inputenc}
\usepackage{xcolor}

\graphicspath{{/home/stephanie/Escritorio/THC/Taller-de-Herramientas-Computacionales/Clases/Latex/Imagenes/}}

\title{\Huge Taller de Herramientas Computacionales}
\author{Stephanie Escobar Sánchez}
\date{24/enero/2019}


\begin{document}
	\maketitle
	\begin{center}
		\includegraphics[scale=0.40]{1.png}	
	\end{center}
	\newpage
	\begin{center}
		\title {\textcolor{orange}{\Huge \textbf{Bitácora clase 14}} }  
	\end{center}

Esta clase la utilizamos principalmente para revisar dudas sobre el laberinto, la mayor parte del tiemplo se empleó en la implementación del programa que pudiera resolver laberintos utilizando primero las listas para representar una matriz true, false y después los comando \textbf{if} y \textbf{else} para realizar las condiciones. Vimos el comando \textbf{elif} que te ahorra un bloque y sirve para poner un if y else al mismo tiempo y simplificar el código \\
La diferencia entre una lista y una tupla es que una tupla es una lista constante, para definirla se usan paréntesis en vez de corchetes, no se puede modificar. La lista si se puede modificar y se usan corchetes.\\
\\
La librería de python numpy tiene métodos de cosas científicas, y se instala por medio del comando \$pip istall,  pip sirve para instalar un modulo de python.
Vimos información sobre cadenas de ADN y cómo cómo contar las letras que se repiten por medio de tres comandos distintos.


	
\end{document}