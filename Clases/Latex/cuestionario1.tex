\documentclass{article}
\usepackage{amsmath}
\usepackage{amssymb}
\usepackage{graphicx}
\usepackage{enumitem}
\usepackage[utf8]{inputenc}
\graphicspath{{/home/stephanie/Escritorio/THC/Taller-de-Herramientas-Computacionales/Clases/Latex/Imagenes/}}

\title{\Huge Taller de Herramientas Computacionales}
\author{Stephanie Escobar Sánchez}
\date{20/enero/2019}
\begin{document}
	\maketitle
\begin{center}
	\includegraphics[scale=0.40]{1.png}	
\end{center}
\newpage
\title{\Huge Cuestionario semana 2} \\
\\
¿Cuál es la diferencia entre visualización y emulación?\\
Una máquina virtual es más rápida que un emulador
debido a que un emulador solo imita lso procesos pero en una maquina virtual
es posible incluso instalar programas del sistema operativo que se virtualiza.\\
\\
¿Qué es el bios?\\
Es una memoria que contiene la informacion de la computadora: Procesador, memoria, características específicas
del equipo.\\
\\
¿Por qué hay distribuciones de linux que corren más rápido que otras?\\
Porque depende de que recursos utilizan en una computadora, es importante ver las características específicas de la computadora para decidir que sistema operativo poner.\\
\\
¿Para qué sirve el comando \$top?\\
Permite ver los núcleos de la computadora para ver la capacidad que tiene.\\
\\
¿Qué es un algoritmo?\\
Un algoritmo es un proceso de instrucciones finitas.\\
\\
¿Por qué una computadora hace aproximaciones?\\
Debido a que es imposible que una computadora tenga a todos los números reales, ya que son infinitos. Por lo que la computadora hace aproximaciones con números muy similares y poco margen de error, pero no podrá ser exacta porque hay un límite de números que nos puede dar.\\
\\
¿Qué es una asignación?\\
Una asignación consiste en asignar valores a una variable a partir de una función o directamente.\\
\\
¿Qué es un bloque en python?\\
Un bloque en python es todo lo que esta indexado, todo lo que esta después de los dos puntos.\\
\\
¿Por qué son importantes los bloques?
que nos permite de una forma visual anidar instrucciones y saber a que bloque corresponde un comando.\\
\\
¿Por qué no es recomendable imprimir el resultado cada vez que se ejecuta un programa?\\
Debido a que muchas veces no necesitamos el resultado hasta el final e imprimirlo cada vez hace que el programa corra mucho más lento que si no lo imprime.\\
\\
¿Para qué es el comando \textit{if}?\\
Es un condicional y permite poner ciertas condiciones que deberían cumplirse para que ejecute una acción, si esto no ocurre, el comando else ejecuta otra acción con el resultado.\\
\\
¿Para qué sirve el comando \textit{while}?
while es un comando que permite crear ciclos y que una acción se repita mientras ciertas condiciones se cumplan, en el momento en que la condición no se cumpla el programa se detiene.\\
\\
¿Cuál es la diferencia entre if y while?\\
El comando if necesita una condición y solo se ejecuta una vez, mientras que el comando while crea ciclos y se ejecuta hasta que se cumpla la condición.\\
\\
¿Qué son los datos de entrada?\\
Son los datos que ya sabemos a partir de los cuales vamos a trabajar.\\
\\
¿Cuáles son los elementos de una función?\\
Se compone de condiciones iniciales y cálculos para llegar a un resultado.\\
\\
¿Qué es lo primero que se debe verificar si se comete algún error?\\
Revisar que no se haya cometido algún error común antes de modificar el código.\\
\\
¿Cuáles son los errores más comunes?\\
Cambiar mayúsculas y minúsculas, errores de dedo, cambio de signos.\\
\\
¿Por qué es importante resolver el problema en lápiz y papel antes de ir directo al código?\\
Porque así nos aseguramos que el resultado es correcto, además de que ya sabemos cómo resolver el problema y ya solo se necesita buscar funciones que nos ayuden.\\
\\
¿Qué es lo primero que se debe poner al probar un programa?\\
Se deben poner cosas triviales que ya sabemos el resultado para ver que lo hace bien, a partir de ello se pueden hacer problemas más complejos.\\
\\
¿Cómo se hace un contador?\\
Nombramos una variable, generalmente la letra i y la igualamos a cero antes del ciclo, de tal manera que irá aumentando cada vez que el ciclo se realiza, lo cual nos permite saber cuantas veces se hizo. \\
\\
¿Cómo se ejecuta un proceso en segundo plano?\\
Poniendo el nombre, un espacio y el símbolo \&\\
\\
¿Cómo se mata un proceso?\\
Con el comando kill -9 o poniendo ctrl+c\\
\\
¿Cómo se ejecuta un programa en python desde la terminal?\\
Poniendo un comentario al principio del código con \#!/usr/bin/python2.7\\
\\
¿Cómo se pueden poner caracteres especiales en un comentario en python?\\
Poniendo el comentario \# -*- coding: utf-8 -*- al principio del documento\\
\\
¿Qué es una clase?\\
Son todos los atributos que comparte un grupo de objetos.\\
\\
¿Qué es un método?\\
Las acciones que realiza un objeto\\
\\
¿Cuál es la notación para llamar a un método en python?\\
objeto.método(el método)\\
\\
¿Cuál es la diferencia entre función y método?\\
Una función no depende de un objeto un método sí\\
\\
¿para qué sirve importar bibliotecas con nombres más cortos?\\
Para no tener que repetir el nombre cada vez que se llama a una función.\\
\\
¿Por qué es recomendable importar todas las funciones de una biblioteca?\\
Porque se pueden confundir con otras funciones al llamarlas. \\
\\
¿Qupe biblioteca de latex nos permite agregar contenido automático en español?\\
Babel\\
\\
¿Cuál es el gestor de paquetes que se utilizó en ubuntu?\\
Synaptic\\
\\
¿Cómo se hace para agregar código e LaTeX?\\
Utilizando en paquete \textit{verbatim}\\
\\
¿Cómo se obtiene el residuo de una división en python?\\
Con \%\\
\\
¿Cómo se le pueden reasignar valores a una variable?\\
Con el nombre de la variable con un += y el número que se quiere sumar para que se sume y se reasigne.\\
\\
¿Para qué sirve la función \textit{bool}? \\
Para evaluar una variable y saber si una variable tiene contenido, si está vacía nos arrojará ”false” y si la variable tiene algo dirá ”true”. También te regresa false cuando el valor de la variable es cero.\\
\\
¿Qué es una lista?\\
Un conjunto de elementos que están guardados en una variable\\
\\
¿Cuál es la diferencia entre índice y posición?\\
El índice es la forma en la que yo accedo a un elemento dentro de la lista la posición es uno más que el índice.\\
\\
¿Para qué sirve \textit{range}?\\
Permite crear una lista con los números de cero hasta uno antes del valor que le ponemos, además permite crear un intervalo y dar un número de número que va avanzando.\\





\end{document}