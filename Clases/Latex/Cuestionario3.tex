\documentclass{article}
\usepackage{amsmath}
\usepackage{amssymb}
\usepackage{graphicx}
\usepackage{enumitem}
\usepackage[utf8]{inputenc}
\usepackage{xcolor}
\graphicspath{{/home/stephanie/Escritorio/THC/Taller-de-Herramientas-Computacionales/Clases/Latex/Imagenes/}}

\title{\Huge Taller de Herramientas Computacionales}
\author{Stephanie Escobar Sánchez}
\date{22/enero/2019}

\begin{document}
	\maketitle
	\begin{center}
		\includegraphics[scale=0.40]{1.png}	
	\end{center}
	\newpage
	\begin{center}
		\title {\textcolor{blue}{\Huge \textbf{Cuestionario clase 12}} }  
	\end{center}
\textbf{¿Qué es una sublista?}\\
Son listas temporales a partir de los elementos de la lista\\
\\
\textbf{¿Cómo se obtiene una sublista?}\\
Con el comando B=A[:], ene l que B es una copia de A\\
\\
\textbf{¿Cuando dos listas son iguales?}\\
Si contienen los mismos elementos y se usa == para igualarlas.\\
\\
\textbf{¿Para qué nos sirve asignar una lista a dos variables?}\\
Para que se guarde una copia modificable.\\
\\
\textbf{¿Qué comando se usa si se quiere recorrer una lista por indices?}\\
 \textit{for i in rage}\\
\\
\textbf{¿Y si se quiere recorrer por sus valores}\\
Con el nombre de la lista \\
\\ 
\textbf{Para qué sirve documentclass{beamer}?}\\
Para hacer presentaciones en latex
\end{document}