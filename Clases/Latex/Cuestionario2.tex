\documentclass{article}
\usepackage{amsmath}
\usepackage{amssymb}
\usepackage{graphicx}
\usepackage{enumitem}
\usepackage[utf8]{inputenc}
\usepackage{xcolor}
\graphicspath{{/home/stephanie/Escritorio/THC/Taller-de-Herramientas-Computacionales/Clases/Latex/Imagenes/}}

\title{\Huge Taller de Herramientas Computacionales}
\author{Stephanie Escobar Sánchez}
\date{21/enero/2019}
\begin{document}
	\maketitle
\begin{center}
	\includegraphics[scale=0.40]{1.png}	
\end{center}
\newpage
\begin{center}
\title {\textcolor{red}{\Huge \textbf{Cuestionario clase 11}} } 
\end{center}
\textbf{¿Para qué sirve el comando \textit{range}?}\\
Para crear una lista automática en un determinado intervalo.\\
\\
\textbf{¿Para qué sirve el comando \textit{len}?}\\
Para contar los elementos de una lista\\
\\
\textbf{¿Qué son los datos de entrada?}
Los datos de entrada son los datos que ya tenemos dados y los traducimos en argumentos de una función\\
\\
\textbf{¿Como son las funciones anidadas}\\
Se utilizan paréntesis y se ejecuta primero la primera opción y después las funciones de afuera.\\
\\
\textbf{¿Para qué se utiliza \textit{enumerate}}
Regresa por cada entrada de una lista el indice que le corresponde y después el valor.\\
\\
\textbf{¿Para que sirve la carpeta .gitignore?}\\
Para que github ignore los archivos que no queremos subir

\end{document}