\documentclass{article}
\usepackage{amsmath}
\usepackage{amssymb}
\usepackage{graphicx}
\usepackage{enumitem}
\usepackage[utf8]{inputenc}
\usepackage{xcolor}


\graphicspath{{/home/stephanie/Escritorio/THC/Taller-de-Herramientas-Computacionales/Clases/Latex/Imagenes/}}

\title{\Huge Taller de Herramientas Computacionales}
\author{Stephanie Escobar Sánchez}
\date{27/enero/2019}


\begin{document}
	\maketitle
	\begin{center}
		\includegraphics[scale=0.40]{1.png}	
	\end{center}
	\newpage
	\begin{center}
		\title {\Huge Problema 1} 
	\end{center}

\textit{Programe una función que determine si dos listas son iguales. Dos listas se
	consideran iguales si tienen igual longitud y sus elementos en cada índice
	también lo son.}\\
\\
Para éste problema lo primero que se considero fue el concepto de igualdad, sabemos que el operador \textbf{==} es utilizado para hacer comparaciones, así que se utilizó  para hacer un condicional y después ya solo escribir si son o no iguales.

\end{document}
