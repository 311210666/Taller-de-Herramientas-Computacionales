\documentclass{article}
\usepackage{amsmath}
\usepackage{amssymb}
\usepackage{graphicx}
\usepackage{enumitem}
\usepackage[utf8]{inputenc}
\usepackage{xcolor}


\graphicspath{{/home/stephanie/Escritorio/THC/Taller-de-Herramientas-Computacionales/Clases/Latex/Imagenes/}}

\title{\Huge Taller de Herramientas Computacionales}
\author{Stephanie Escobar Sánchez}
\date{23/enero/2019}


\begin{document}
	\maketitle
	\begin{center}
		\includegraphics[scale=0.40]{1.png}	
	\end{center}
	\newpage
	\begin{center}
		\title {\Huge Problema 3} 
	\end{center}

\textit{El problema consistió en convertir F a C y viceversa}\\
\\

El este problema lo primero que se tuvo que analizar fue cómo se transformaban los grados manualmente, una vez teniendo la fórmula solo se daban las variables y listo. Para convertir los grados que el usuario quiera se agregó un \textit{input()} en el script.

\section*{Listas}

Para la parte de las listas fue necesario primero definir un intervalo en el cual queriamos trabajar, posteriormente se relacionó el primer intervalo hecho con la fórmula para transformar, así se creó una tabla de conversiones en un intervalo dado.


\end{document}