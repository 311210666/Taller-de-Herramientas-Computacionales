\documentclass{article}
\usepackage{amsmath}
\usepackage{amssymb}
\usepackage{graphicx}
\usepackage{enumitem}
\usepackage[utf8]{inputenc}
\usepackage{xcolor}


\graphicspath{{/home/stephanie/Escritorio/THC/Taller-de-Herramientas-Computacionales/Clases/Latex/Imagenes/}}

\title{\Huge Taller de Herramientas Computacionales}
\author{Stephanie Escobar Sánchez}
\date{23/enero/2019}


\begin{document}
	\maketitle
	\begin{center}
		\includegraphics[scale=0.40]{1.png}	
	\end{center}
	\newpage
	\begin{center}
		\title {\Huge Problema 4} 
	\end{center}

\textit{Calcular el n-esimo término de la suc de fibonacci con n natural y 0}\\
\\
En un principio parecía difícil, pero al analizar el problema sin programar nada se tomo en cuenta el patrón, cada termino es la suma de los dos términos anteriores, al notar esto al principio se intentó hacer directo pero no funciono, comencé haciendo las condiciones en caso de que fueran los primeros y posteriormente se hizo la función llamándose a si misma.\\
\\
\section*{Listas}
Para realizar este problema con listas lo primero fue utilizar la función \textit{range} para crear una lista  y posteriormente otra que llamara a la lista anterior. Así nos aprecian el n-enésimo término y los anteriores

\end{document}