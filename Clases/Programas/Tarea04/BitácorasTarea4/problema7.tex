\documentclass{article}
\usepackage{amsmath}
\usepackage{amssymb}
\usepackage{graphicx}
\usepackage{enumitem}
\usepackage[utf8]{inputenc}
\usepackage{xcolor}


\graphicspath{{/home/stephanie/Escritorio/THC/Taller-de-Herramientas-Computacionales/Clases/Latex/Imagenes/}}

\title{\Huge Taller de Herramientas Computacionales}
\author{Stephanie Escobar Sánchez}
\date{23/enero/2019}


\begin{document}
	\maketitle
	\begin{center}
		\includegraphics[scale=0.40]{1.png}	
	\end{center}
	\newpage
	\begin{center}
		\title {\Huge Problema 7} 
	\end{center}

\textit{Calcular el promedio de 10 datos, el mayor número y el menor}\\
\\
En este caso se utilizo el promedio que ya se había hecho anteriormente, al principio fue difícil obtener el mayor y el menor de los número debían compararse todos, así que agregué 10 lineas en las que comparaba a un número con todos los demás, así hasta obtener un código muy largo pero funcional y no supe después como hacerlo más pequeño.\\
\\
\section*{Listas}
Para el caso de las listas no supe como aplicarlas debido a que el un principio hice una función complicada que no pude mejorar y no era compatible con las listas.

\end{document} 
