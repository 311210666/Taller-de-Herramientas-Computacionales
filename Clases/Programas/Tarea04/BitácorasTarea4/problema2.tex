\documentclass{article}
\usepackage{amsmath}
\usepackage{amssymb}
\usepackage{graphicx}
\usepackage{enumitem}
\usepackage[utf8]{inputenc}
\usepackage{xcolor}

\graphicspath{{/home/stephanie/Escritorio/THC/Taller-de-Herramientas-Computacionales/Clases/Latex/Imagenes/}}

\title{\Huge Taller de Herramientas Computacionales}
\author{Stephanie Escobar Sánchez}
\date{23/enero/2019}


\begin{document}
	\maketitle
	\begin{center}
		\includegraphics[scale=0.40]{1.png}	
	\end{center}
\newpage
	\begin{center}
		\title {\Huge Problema 2} 
	\end{center}
\textit{Este problema consistió en determinar en qué tiempos se alcanza una altura en específico.}\\
\\
Éste problema fue resuelto de una manera muy rápida debido a que ya habíamos hecho uno en clase en donde calculamos la altura, esto facilitó el proceso, ya que solo se trató de despejar y posteriormente nombrar variables y aplicar la fórmula ya dada.

\section*{Listas}

Para pasar al problema con listas lo unico que cambió es que ésta vez las variables fueron dadas en una lista, así se pudieron meter al problema normalmente.

\end{document}