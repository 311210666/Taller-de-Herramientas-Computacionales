\documentclass{article}
\usepackage{amsmath}
\usepackage{amssymb}
\usepackage{graphicx}
\usepackage{enumitem}
\usepackage[utf8]{inputenc}
\usepackage{xcolor}


\graphicspath{{/home/stephanie/Escritorio/THC/Taller-de-Herramientas-Computacionales/Clases/Latex/Imagenes/}}

\title{\Huge Taller de Herramientas Computacionales}
\author{Stephanie Escobar Sánchez}
\date{23/enero/2019}


\begin{document}
	\maketitle
	\begin{center}
		\includegraphics[scale=0.40]{1.png}	
	\end{center}
	\newpage
	\begin{center}
		\title {\Huge Problema 5} 
	\end{center}

\textit{Calcular la suma de los primeros n naturales}\\
\\
Este problema no fue difícil, ya que únicamente se utilizó la fórmula de Euler para obtenerlo. Así que solo se asignaron variables y se ejecutó la fórmula.\\
\\
\section{Listas}
Para esta sección solo se hizo una lista a la que se le fueron agregando elementos con la función \textit{for i in range} y después realizar la suma de cada uno, fue menos fácil pero más completo.
\end{document}